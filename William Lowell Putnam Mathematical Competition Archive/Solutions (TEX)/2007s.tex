\documentclass[amssymb,twocolumn,pra,10pt,aps]{revtex4-1}
\usepackage{mathptmx,amsmath}

\newcommand{\RR}{\mathbb{R}}
\newcommand{\CC}{\mathbb{C}}
\DeclareMathOperator{\lcm}{lcm}
\DeclareMathOperator{\sgn}{sgn}

\begin{document}
\title{Solutions to the 68th William Lowell Putnam Mathematical Competition \\
    Saturday, December 1, 2007}
\author{Manjul Bhargava, Kiran Kedlaya, and Lenny Ng}
\noaffiliation
\maketitle

\begin{itemize}

\item[A--1]
The only such $\alpha$ are $2/3, 3/2, (13 \pm \sqrt{601})/12$.

\textbf{First solution:}
Let $C_1$ and $C_2$ be the curves
$y=\alpha x^2 + \alpha x + \frac{1}{24}$
and $x=\alpha y^2 + \alpha y + \frac{1}{24}$, respectively,
and let $L$ be the line $y=x$.
We consider three cases.

If $C_1$ is tangent to $L$, then the point of tangency $(x,x)$ satisfies
\[
2\alpha x + \alpha = 1, \qquad x = \alpha x^2 + \alpha x + \frac{1}{24};
\]
by symmetry, $C_2$ is tangent to $L$ there, so $C_1$ and $C_2$ are tangent.
Writing $\alpha = 1/(2x+1)$ in the first equation and substituting into
the second, we must have
\[
x = \frac{x^2+x}{2x+1} + \frac{1}{24},
\]
which simplifies to $0 = 24x^2 - 2x - 1
= (6x+1)(4x-1)$, or $x \in \{1/4, -1/6\}$. This yields
$\alpha = 1/(2x+1) \in \{2/3, 3/2\}$.

If $C_1$ does not intersect $L$, then $C_1$ and $C_2$ are separated by $L$
and so cannot be tangent.

If $C_1$ intersects $L$ in two distinct points $P_1, P_2$, then it is not
tangent to $L$ at either point. Suppose at one of these points, say $P_1$,
the tangent to $C_1$ is perpendicular to $L$; then by symmetry, the same
will be true of $C_2$, so $C_1$ and $C_2$ will be tangent at $P_1$. In this
case, the point $P_1 = (x,x)$ satisfies
\[
2 \alpha x + \alpha = -1, \qquad x = \alpha x^2 + \alpha x + \frac{1}{24};
\]
writing $\alpha = -1/(2x+1)$ in the first equation and substituting into
the second, we have
\[
x = -\frac{x^2+x}{2x+1} + \frac{1}{24},
\]
or
$x = (-23 \pm \sqrt{601})/72$.
This yields
$\alpha = -1/(2x+1) = (13 \pm \sqrt{601})/12$.

If instead the tangents to $C_1$ at $P_1, P_2$ are not perpendicular to
$L$, then we claim there cannot be any point where $C_1$ and $C_2$ are tangent.
Indeed, if we count intersections of $C_1$ and $C_2$ (by using $C_1$ to
substitute for $y$ in $C_2$, then solving for $y$), we get at most four
solutions counting multiplicity. Two of these are $P_1$ and $P_2$,
and any point of tangency counts for two more. However, off of $L$,
any point of tangency would have a mirror image which is also a point of
tangency, and there cannot be six solutions. Hence we have now found all
possible $\alpha$.

\textbf{Second solution:}
For any nonzero value of $\alpha$, the two
conics will intersect in four points in the complex projective plane
$\mathbb{P}^2(\mathbb{C})$. To determine the
$y$-coordinates of these intersection points, subtract the two equations
to obtain
\[
(y-x) = \alpha(x-y)(x+y) + \alpha(x-y).
\]
Therefore, at
a point of intersection we have either $x=y$, or $x = -1/\alpha - (y+1)$.
Substituting these two possible linear conditions into the second
equation shows that the $y$-coordinate of a point of intersection is a
root of either
$Q_1(y) = \alpha y^2+(\alpha-1)y + 1/24$ or
$Q_2(y) = \alpha y^2 + (\alpha+1) y + 25/24 +1/\alpha$.

If two curves
are tangent, then the $y$-coordinates of at least two of the
intersection points will coincide; the converse is also true because one of the
curves is the graph of a function in $x$. The coincidence
occurs precisely when either
the discriminant of at least one of $Q_1$ or $Q_2$ is zero, or
there is a common root of $Q_1$ and $Q_2$. Computing the discriminants of $Q_1$
and $Q_2$ yields (up to constant factors) $f_1(\alpha)=6\alpha^2 -
13\alpha + 6$ and $f_2(\alpha)=6\alpha^2 - 13\alpha - 18$, respectively.
If on the other hand $Q_1$ and $Q_2$ have a common root, it must
be also a root of $Q_2(y) - Q_1(y) = 2y +1 + 1/\alpha$,
yielding $y = -(1+\alpha)/(2\alpha)$ and
$0 = Q_1(y) = -f_2(\alpha)/(24 \alpha)$.

Thus the values of
$\alpha$ for which the two curves are tangent must be contained in the
set of zeros of $f_1$ and $f_2$, namely $2/3$, $3/2$, and
$(13\pm\sqrt{601})/12$.

\textbf{Remark:}
The fact that the two conics in $\mathbb{P}^2(\CC)$ meet in four points, counted
with multiplicities, is a special case of \emph{B\'ezout's theorem}: two
curves in $\mathbb{P}^2(\CC)$ of degrees $m, n$ and not sharing any common component
meet in exactly $mn$ points when counted with multiplicity.

Many solvers were surprised that the proposers chose the parameter $1/24$
to give two rational roots and two nonrational roots. In fact, they had
no choice in the matter: attempting to make all four roots rational
by replacing $1/24$ by $\beta$ amounts
to asking for $\beta^2 + \beta$ and $\beta^2 + \beta + 1$ to be perfect
squares. This cannot happen outside of trivial cases ($\beta = 0, -1$)
ultimately because the elliptic curve 24A1 (in Cremona's notation)
over $\mathbb{Q}$ has rank $0$. (Thanks to Noam Elkies for providing this
computation.)

However, there are choices that make the radical milder,
e.g., $\beta = 1/3$  gives
$\beta^2 + \beta = 4/9$ and $\beta^2 + \beta + 1 = 13/9$,
while $\beta = 3/5$ gives $\beta^2 + \beta = 24/25$
and $\beta^2 + \beta + 1 = 49/25$.

\item[A--2]
The minimum is 4, achieved by the square with vertices $(\pm 1, \pm 1)$.

\textbf{First solution:}
To prove that 4 is a lower bound, let $S$ be a convex set of the desired
form. Choose $A,B,C,D \in S$ lying on the branches of the two hyperbolas,
with $A$ in the upper right
quadrant, $B$ in the upper left, $C$ in the lower left, $D$ in the lower right.
Then the area of the quadrilateral $ABCD$ is a lower bound for
the area of $S$.

Write $A = (a,1/a)$,
$B = (-b,1/b)$, $C = (-c,-1/c)$, $D = (d, -1/d)$ with $a,b,c,d > 0$.
Then the area of the quadrilateral $ABCD$ is
\[
\frac{1}{2}(a/b + b/c + c/d + d/a + b/a + c/b + d/c + a/d),
\]
which by the arithmetic-geometric mean inequality is at least
$4$.

\textbf{Second solution:}
Choose $A,B,C,D$ as in the first solution.
Note that both the hyperbolas and the area of the convex hull of $ABCD$ are
invariant under the transformation $(x,y) \mapsto (xm, y/m)$ for any
$m>0$. For $m$ small, the counterclockwise angle from the line $AC$ to
the line $BD$ approaches 0; for $m$ large, this angle approaches $\pi$.
By continuity, for some $m$ this angle becomes $\pi/2$, that is,
$AC$ and $BD$ become perpendicular. The area of $ABCD$
is then $AC \cdot BD$.

It thus suffices to note that $AC \geq 2 \sqrt{2}$ (and similarly for $BD$).
This holds because if we draw the tangent lines to the hyperbola $xy=1$
at the points $(1,1)$ and $(-1,-1)$, then $A$ and $C$ lie outside the region
between these lines. If we project the segment $AC$ orthogonally
onto the line $x=y=1$, the resulting projection has length at
least $2 \sqrt{2}$,
so $AC$ must as well.

\textbf{Third solution:}
(by Richard Stanley)
Choose $A,B,C,D$ as in the first solution. Now fixing $A$ and $C$, move $B$ and
$D$ to the points at which the tangents to the curve are parallel to the line
$AC$. This does not increase the area of the quadrilateral $ABCD$ (even if
this quadrilateral is not convex).

Note that $B$ and $D$ are now diametrically opposite; write $B = (-x, 1/x)$
and $D = (x, -1/x)$. If we thus repeat the
procedure, fixing $B$ and $D$ and moving $A$ and $C$ to the points where the
tangents are parallel to $BD$, then $A$ and $C$ must move to $(x, 1/x)$
and $(-x,-1/x)$, respectively, forming a rectangle of area 4.

\textbf{Remark:}
Many geometric solutions are possible. An example suggested by David Savitt
(due to Chris Brewer):
note that $AD$ and $BC$ cross the
positive and negative $x$-axes, respectively, so the convex hull of $ABCD$
contains $O$. Then check that the area of triangle $OAB$ is at least 1, et cetera.


\item[A--3]
Assume that we have an ordering of $1,2,\dots,3k+1$ such
  that no initial subsequence sums to $0$ mod $3$. If we omit the
  multiples of $3$ from this ordering, then the remaining sequence mod
  $3$ must look like $1,1,-1,1,-1,\ldots$ or $-1,-1,1,-1,1,\ldots$.
  Since there is one more integer in the ordering congruent to $1$ mod
  $3$ than to $-1$, the sequence mod $3$ must look like
  $1,1,-1,1,-1,\ldots$.

  It follows that the ordering satisfies the given condition if and
  only if the following two conditions hold: the first element in the
  ordering is not divisible by $3$, and the sequence mod $3$ (ignoring
zeroes) is of the form $1,1,-1,1,-1,\ldots$. The two conditions are
  independent, and the probability of the first is $(2k+1)/(3k+1)$
while the probability of the second is
$1/\binom{2k+1}{k}$,
  since there are $\binom{2k+1}{k}$ ways to order $(k+1)$ $1$'s and $k$
  $-1$'s.
  Hence the desired probability is the product of these two, or
  $\frac{k!(k+1)!}{(3k+1)(2k)!}$.

\item[A--4]
Note that $n$ is a repunit if and only if $9n+1 = 10^m$ for some power of
10 greater than 1. Consequently, if we put
\[
g(n) = 9f\left( \frac{n-1}{9} \right) + 1,
\]
then $f$ takes repunits to repunits if and only if $g$ takes powers of 10
greater than 1 to powers of 10 greater than 1. We will show that the only
such functions $g$ are those of the form $g(n) = 10^c n^d$ for $d \geq 0$,
$c \geq 1-d$ (all of which clearly work),
which will mean that the desired polynomials $f$ are those of the form
\[
f(n) = \frac{1}{9}(10^c (9n+1)^d - 1)
\]
for the same $c,d$.

It is convenient to allow ``powers of 10'' to be of the form
$10^k$ for any integer $k$. With this convention, it suffices to check that
the polynomials $g$ taking powers of 10 greater than 1 to powers of 10
are of the form $10^c n^d$ for any integers $c,d$ with $d \geq 0$.

\textbf{First solution:}
Suppose that the leading term of $g(x)$ is $ax^d$, and note that
$a>0$. As $x \to \infty$, we have $g(x)/x^d \to a$; however, for $x$
a power of 10 greater than 1, $g(x)/x^d$ is a power of 10.
The set of powers of 10 has no positive limit point, so $g(x)/x^d$
must be equal to $a$ for $x = 10^k$ with $k$ sufficiently large,
and we must have $a = 10^c$ for some $c$.
The polynomial $g(x) - 10^c x^d$ has infinitely many roots, so
must be identically zero.

\textbf{Second solution:}
We proceed by induction on $d = \deg(g)$.
If $d=0$, we have $g(n) = 10^c$ for some $c$. Otherwise,
$g$ has rational coefficients by Lagrange's interpolation formula (this
applies to any polynomial of degree $d$ taking at least $d+1$
different rational numbers to rational numbers), so $g(0) = t$ is rational.
Moreover, $g$ takes each value only finitely many times, so the sequence
$g(10^0), g(10^1), \dots$ includes arbitrarily large powers of 10.
Suppose that $t \neq 0$; then we can choose a positive integer $h$ such that
the numerator of $t$ is not divisible by $10^h$.
But for $c$ large enough, $g(10^c) - t$ has numerator divisible by
$10^b$ for some $b>h$, contradiction.

Consequently, $t=0$, and we may apply the induction hypothesis to $g(n)/n$
to deduce the claim.

\textbf{Remark:} The second solution amounts to the fact that $g$, being
a polynomial with rational coefficients, is continuous for the $2$-adic
and $5$-adic topologies on $\mathbb{Q}$. By contrast, the first solution
uses the ``$\infty$-adic'' topology, i.e., the usual real topology.

\item[A--5]
In all solutions, let $G$ be a finite group of order $m$.

\textbf{First solution:}
By Lagrange's theorem, if $m$ is
not divisible by $p$, then $n = 0$. Otherwise, let $S$ be the set of
$p$-tuples $(a_0,\dots,a_{p-1}) \in G^p$ such that $a_0 \cdots a_{p-1} = e$;
then $S$ has cardinality $m^{p-1}$, which is divisible by $p$.
Note that this set is invariant under cyclic permutation, that is,
if $(a_0,\dots,a_{p-1}) \in S$, then $(a_1,\dots,a_{p-1},a_0) \in S$ also.
The fixed points under this operation are the tuples
$(a,\dots,a)$ with $a^p = e$; all other tuples can be grouped into orbits
under cyclic permutation, each of which has size $p$. Consequently,
the number of $a \in G$ with $a^p = e$ is divisible by $p$; since that number
is $n+1$ (only $e$ has order 1), this proves the claim.

\textbf{Second solution:}
(by Anand Deopurkar)
Assume that $n > 0$, and let $H$ be any subgroup of $G$ of order $p$.
Let $S$ be the set of all elements of $G \setminus H$
of order dividing $p$, and let
$H$ act on $G$ by conjugation. Each orbit has size $p$ except for those
which consist of individual elements $g$ which commute with $H$.
For each such $g$, $g$ and $H$ generate an elementary abelian subgroup of
$G$ of order $p^2$. However, we can group these $g$ into sets of size
$p^2-p$ based on which subgroup they generate together with $H$.
Hence the cardinality of $S$ is divisible by $p$; adding the $p-1$
nontrivial elements of $H$ gives $n \equiv -1 \pmod{p}$ as desired.

\textbf{Third solution:}
Let $S$ be the set of elements in $G$ having order dividing $p$, and let
$H$ be an elementary abelian $p$-group of maximal order in $G$.  If
$|H|=1$, then we are done.  So assume $|H|=p^k$ for some $k\geq 1$, and
let $H$ act on $S$ by conjugation.  Let $T\subset S$ denote the set of
fixed points of this action.  Then the size of every $H$-orbit on $S$
divides $p^k$, and so $|S|\equiv |T| \pmod{p}$.
On the other hand, $H\subset T$, and if $T$ contained an element not in $H$,
then that would contradict the maximality of $H$.  It follows that
$H=T$, and so $|S|\equiv |T| = |H| = p^k \equiv 0 \pmod{p}$,
i.e., $|S|=n+1$ is a multiple of $p$.

\textbf{Remark:} This result is a theorem of Cauchy; the first solution
above is due to McKay. A more general (and more difficult) result was proved
by Frobenius: for any positive integer $m$, if $G$ is a finite group of order
divisible by $m$, then the number of elements of $G$ of order dividing $m$
is a multiple of $m$.

\item[A--6]
For an admissible triangulation $\mathcal{T}$, number the
  vertices of $P$ consecutively $v_1,\dots,v_n$, and let $a_i$ be the
  number of edges in $\mathcal{T}$ emanating from $v_i$; note that
  $a_i \geq 2$ for all $i$.

  We first claim that $a_1+\dots+a_n \leq 4n-6$. Let $V,E,F$ denote
  the number of vertices, edges, and faces in $\mathcal{T}$. By
  Euler's Formula, $(F+1)-E+V=2$ (one must add 1 to the face count for the
  region exterior to $P$). Each face has three edges, and each edge but the
  $n$ outside edges belongs to two faces; hence $F = 2E-n$. On the
  other hand, each edge has two endpoints, and each of the $V-n$
  internal vertices is an endpoint of at least $6$ edges; hence
  $a_1+\dots+a_n+6(V-n) \leq 2E$. Combining this inequality with the
  previous two equations gives
\begin{align*}
a_1+\dots+a_n &\leq 2E+6n-6(1-F+E) \\
&= 4n-6,
\end{align*}
as claimed.

Now set $A_3 = 1$ and $A_n = A_{n-1}+2n-3$ for $n \geq 4$; we will
prove by induction on $n$ that $\mathcal{T}$ has at most $A_n$
triangles. For $n=3$, since $a_1+a_2+a_3=6$, $a_1=a_2=a_3=2$ and hence
$\mathcal{T}$ consists of just one triangle.

Next assume that an admissible triangulation of an $(n-1)$-gon has at
most $A_{n-1}$ triangles, and let $\mathcal{T}$ be an admissible
triangulation of an $n$-gon. If any $a_i = 2$, then we can remove the
triangle of $\mathcal{T}$ containing vertex $v_i$ to obtain an
admissible triangulation of an $(n-1)$-gon; then the number of
triangles in $\mathcal{T}$ is at most $A_{n-1}+1 < A_n$ by induction.
Otherwise, all $a_i \geq 3$. Now the average of $a_1,\dots,a_n$ is
less than $4$, and thus there are more $a_i = 3$ than $a_i \geq 5$. It
follows that there is a sequence of $k$ consecutive vertices in $P$
whose degrees are $3,4,4,\ldots,4,3$ in order, for some $k$ with
$2\leq k\leq n-1$ (possibly $k=2$, in which case there are no degree
$4$ vertices separating the degree $3$ vertices). If we remove from
$\mathcal{T}$ the $2k-1$ triangles which contain at least one of these
vertices, then we are left with an admissible triangulation of an
$(n-1)$-gon. It follows that there are at most $A_{n-1}+2k-1 \leq
A_{n-1}+2n-3 = A_n$ triangles in $\mathcal{T}$. This completes the
induction step and the proof.

\textbf{Remark:}
We can refine the bound $A_n$ somewhat.
Supposing that $a_i \geq 3$ for all
$i$, the fact that $a_1 + \cdots + a_n \leq 4n-6$ implies that there are at least
six more indices $i$ with $a_i = 3$ than with $a_i \geq 5$. Thus there exist
six sequences with degrees $3,4,\dots,4,3$, of total length at most
$n+6$. We may thus choose a sequence of length $k \leq \lfloor \frac{n}{6}
\rfloor + 1$, so we may improve the upper bound to $A_n = A_{n-1} +
2 \lfloor \frac{n}{6} \rfloor + 1$, or asymptotically
$\frac{1}{6} n^2$.

However (as noted by Noam Elkies), a hexagonal swatch of a triangular
lattice, with the boundary as close to regular as possible,
achieves asymptotically $\frac{1}{6} n^2$ triangles.

\item[B--1]
The problem fails if $f$ is allowed to be constant, e.g., take $f(n) = 1$.
We thus assume that $f$ is nonconstant.
Write $f(n) = \sum_{i=0}^d a_i n^i$ with $a_i > 0$. Then
\begin{align*}
f(f(n)+1) &= \sum_{i=0}^d a_i (f(n) + 1)^i \\
&\equiv f(1) \pmod{f(n)}.
\end{align*}
If $n = 1$, then this implies that $f(f(n)+1)$ is divisible by $f(n)$.
Otherwise, $0 < f(1) < f(n)$ since $f$ is nonconstant and has positive
coefficients, so $f(f(n)+1)$ cannot be divisible by $f(n)$.

\item[B--2]
Put $B = \max_{0 \leq x \leq 1} |f'(x)|$
and $g(x) = \int_0^x f(y)\,dy$. Since $g(0) = g(1) = 0$, the maximum value
of $|g(x)|$ must occur at a critical point $y \in (0,1)$ satisfying
$g'(y) = f(y) = 0$. We may thus take $\alpha = y$ hereafter.

Since $\int_0^\alpha f(x)\,dx = -\int_0^{1-\alpha} f(1-x)\,dx$,
we may assume that $\alpha \leq 1/2$. By then substituting $-f(x)$
for $f(x)$ if needed, we may assume that $\int_0^\alpha f(x)\,dx \geq 0$.
{}From the inequality $f'(x) \geq -B$, we deduce
$f(x) \leq B(\alpha - x)$ for $0 \leq x \leq \alpha$, so
\begin{align*}
\int_0^\alpha f(x)\,dx \leq &\int_0^\alpha B(\alpha-x)\,dx \\
&= -\left. \frac{1}{2} B (\alpha - x)^2 \right|_0^\alpha \\
&= \frac{\alpha^2}{2} B \leq \frac{1}{8} B
\end{align*}
as desired.

\item[B--3]
\textbf{First solution:}
Observing that $x_2/2 = 13$, $x_3/4=34$, $x_4/8=89$, we
guess that $x_n = 2^{n-1} F_{2n+3}$, where $F_k$ is the $k$-th
Fibonacci number. Thus we claim that $x_n = \frac{2^{n-1}}{\sqrt{5}}
(\alpha^{2n+3}-\alpha^{-(2n+3)})$, where $\alpha =
\frac{1+\sqrt{5}}{2}$, to make the answer $x_{2007} =
\frac{2^{2006}}{\sqrt{5}}(\alpha^{3997}-\alpha^{-3997})$.

We prove the claim by induction; the base case $x_0 = 1$ is true, and
so it suffices to show that the recursion $x_{n+1} = 3x_n + \lfloor
x_n \sqrt{5} \rfloor$ is satisfied for our formula for $x_n$. Indeed,
since $\alpha^2 = \frac{3+\sqrt{5}}{2}$, we have
\begin{align*}
x_{n+1}-(3+\sqrt{5})x_n &= \frac{2^{n-1}}{\sqrt{5}}
(2(\alpha^{2n+5}-\alpha^{-(2n+5)}) \\
&\quad-(3+\sqrt{5})(\alpha^{2n+3}-\alpha^{-(2n+3)})) \\
&= 2^n \alpha^{-(2n+3)}.
\end{align*}
Now $2^n \alpha^{-(2n+3)} =
(\frac{1-\sqrt{5}}{2})^3 (3-\sqrt{5})^n$ is between $-1$ and $0$; the
recursion follows since $x_n,x_{n+1}$ are integers.

\textbf{Second solution:}
(by Catalin Zara)
Since $x_n$ is rational, we have $0 < x_n \sqrt{5} - \lfloor x_n \sqrt{5}
\rfloor < 1$. We now have the inequalities
\begin{gather*}
x_{n+1}-3x_n < x_n \sqrt{5} < x_{n+1} -3x_n+1 \\
(3+\sqrt{5})x_n - 1 < x_{n+1} < (3+\sqrt{5})x_n \\
4x_n - (3-\sqrt{5}) < (3-\sqrt{5})x_{n+1} < 4x_n \\
3x_{n+1} - 4x_n < x_{n+1} \sqrt{5} < 3x_{n+1}-4x_n + (3-\sqrt{5}).
\end{gather*}
Since $0 < 3-\sqrt{5} < 1$, this yields $\lfloor x_{n+1} \sqrt{5}
\rfloor = 3x_{n+1} - 4x_n$, so we can rewrite the recursion as
$x_{n+1} = 6x_n - 4x_{n-1}$ for $n \geq 2$. It is routine to solve
this recursion to obtain the same solution as above.

\textbf{Remark:}
With an initial 1 prepended,
this becomes
sequence A018903 in Sloane's On-Line Encyclopedia of Integer Sequences:
(\texttt{http://www.research.att.com/\~{}njas/ sequences/}).
Therein, the sequence is described as the case $S(1,5)$ of the sequence
$S(a_0, a_1)$ in which $a_{n+2}$ is the least integer for which
$a_{n+2}/a_{n+1}>a_{n+1}/a_n$. Sloane cites
 D. W. Boyd, Linear recurrence relations for some generalized Pisot sequences,
\textit{Advances in Number Theory (Kingston, ON, 1991)}, Oxford Univ.
Press, New York, 1993, p.\ 333--340.

\item[B--4]
The number of pairs is $2^{n+1}$. The degree condition forces
$P$ to have degree $n$ and leading coefficient $\pm 1$; we may count
pairs in which $P$ has leading coefficient 1 as long as we multiply by $2$
afterward.

Factor both sides:
\begin{align*}
& (P(X) + Q(X)i)(P(X) - Q(X)i) \\
& = \prod_{j=0}^{n-1}
(X - \exp(2 \pi i (2j+1)/(4n))) \\
& \quad \cdot \prod_{j=0}^{n-1}
(X + \exp(2 \pi i (2j+1)/(4n))).
\end{align*}
Then each choice of $P,Q$ corresponds to equating $P(X) + Q(X)i$
with the product of some $n$ factors on the right,
in which we choose exactly of the two factors for each $j=0,\dots,n-1$.
(We must take exactly $n$
factors because as a polynomial in $X$ with complex coefficients, $P(X) + Q(X)i$
has degree exactly $n$. We must choose one for each $j$ to ensure that
$P(X) + Q(X)i$ and $P(X) -Q(X)i$ are complex conjugates, so that $P, Q$ have
real coefficients.) Thus there are $2^n$ such pairs;
multiplying by 2 to allow $P$ to have
leading coefficient $-1$ yields the desired result.

\textbf{Remark:} If we allow $P$ and $Q$ to have complex coefficients but
still require $\deg(P) > \deg(Q)$, then the number of pairs increases
to $2\binom{2n}{n}$, as we may choose any $n$ of the $2n$ factors of
$X^{2n}+1$ to use to form $P(X) + Q(X)i$.

\item[B--5]
  For $n$ an integer, we have
$\left\lfloor \frac{n}{k} \right\rfloor = \frac{n-j}{k}$ for $j$
the unique integer in $\{0,\dots,k-1\}$ congruent to $n$ modulo $k$;
hence
\[
\prod_{j=0}^{k-1} \left( \left\lfloor \frac{n}{k} \right\rfloor - \frac{n-j}{k}
\right) = 0.
\]
By expanding this out, we obtain
the desired polynomials $P_0(n), \dots, P_{k-1}(n)$.

\textbf{Remark:} Variants of this solution are possible that
construct the $P_i$
less explicitly, using Lagrange interpolation or Vandermonde determinants.

\item[B--6]
(Suggested by Oleg Golberg)
Assume $n \geq 2$, or else the problem
is trivially false.
Throughout this proof, any $C_i$ will be a positive constant whose exact
value is immaterial.
As in the proof of Stirling's approximation, we
estimate for any fixed $c \in \RR$,
\[
\sum_{i=1}^n (i+c) \log i =
\frac{1}{2} n^2 \log n - \frac{1}{4} n^2 + O(n \log n)
\]
by comparing the sum to an integral. This gives
\begin{align*}
n^{n^2/2-C_1 n} e^{-n^2/4} &\leq 1^{1+c} 2^{2+c} \cdots n^{n+c} \\
&\leq n^{n^2/2+C_2 n} e^{-n^2/4}.
\end{align*}
We now interpret $f(n)$ as counting the number of $n$-tuples
$(a_1, \dots, a_n)$ of nonnegative integers such that
\[
a_1 1! + \cdots + a_n n! = n!.
\]
For an upper bound on $f(n)$, we use the inequalities
$0 \leq a_i \leq n!/i!$ to deduce that there are at most $n!/i! + 1 \leq 2(n!/i!)$
choices for $a_i$. Hence
\begin{align*}
f(n) &\leq 2^n \frac{n!}{1!} \cdots \frac{n!}{n!} \\
&= 2^n 2^1 3^2 \cdots n^{n-1} \\
&\leq n^{n^2/2+C_3 n} e^{-n^2/4}.
\end{align*}
For a lower bound on $f(n)$, we note that if $0 \leq a_i < (n-1)!/i!$
for $i=2,\dots,n-1$ and $a_n = 0$,
then $0 \leq a_2 2! + \cdots + a_n n! \leq n!$, so there is a unique
choice of $a_1$ to complete this to a solution of
$a_1 1! + \cdots + a_n n! = n!$. Hence
\begin{align*}
f(n) &\geq \frac{(n-1)!}{2!} \cdots \frac{(n-1)!}{(n-1)!} \\
&= 3^1 4^2 \cdots (n-1)^{n-3} \\
&\geq n^{n^2/2+C_4 n} e^{-n^2/4}.
\end{align*}

\end{itemize}
\end{document}



