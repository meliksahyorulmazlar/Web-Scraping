\documentclass[amssymb,twocolumn,pra,10pt,aps]{revtex4-1}
\usepackage{mathptmx,amsmath}

\begin{document}
\title{Solutions to the 56th William Lowell Putnam Mathematical Competition \\
    Saturday, December 2, 1995}
\author{Kiran Kedlaya}
\noaffiliation
\maketitle

\begin{itemize}
\item[A--1]
Suppose on the contrary that there exist $t_{1}, t_{2} \in T$
with $t_{1}t_{2} \in U$ and $u_{1}, u_{2} \in U$ with $u_{1}u_{2} \in
T$. Then $(t_{1}t_{2})u_{1}u_{2} \in U$ while
$t_{1}t_{2}(u_{1}u_{2}) \in T$, contradiction.

\item[A--2]
The integral converges iff $a=b$. The easiest proof uses
``big-O'' notation and the fact that $(1+x)^{1/2} = 1 + x/2 +
O(x^{2})$ for $|x|<1$. (Here $O(x^{2})$ means bounded by a constant
times $x^{2}$.)

So
\begin{align*}
\sqrt{x+a}-\sqrt{x} &= x^{1/2}(\sqrt{1+a/x} - 1) \\
&= x^{1/2}(1 + a/2x + O(x^{-2})),
\end{align*}
hence
\[
\sqrt{\sqrt{x+a} - \sqrt{x}} = x^{1/4} (a/4x + O(x^{-2}))
\]
and similarly
\[
\sqrt{\sqrt{x} - \sqrt{x-b}} = x^{1/4} (b/4x + O(x^{-2})).
\]
Hence the integral we're looking at is
\[
\int_{b}^{\infty} x^{1/4} ((a-b)/4x + O(x^{-2}))\,dx.
\]
The term $x^{1/4} O(x^{-2})$ is bounded by a constant times
$x^{-7/4}$, whose integral converges. Thus we only have to decide
whether $x^{-3/4} (a-b)/4$ converges. But $x^{-3/4}$ has divergent
integral, so we get convergence if and only if $a=b$ (in which case
the integral telescopes anyway).

\item[A--3]
Let $D$ and $E$ be the numbers $d_{1}\dots d_{9}$ and $e_{1}\dots
e_{9}$, respectively. We are given that $(e_{i} - d_{i})10^{9-i} + D
\equiv 0 \pmod 7$ and $(f_{i} - e_{i})10^{9-i} + E \equiv 0 \pmod 7$
for $i=1, \dots, 9$. Sum the first relation over $i=1,\dots,9$ and we
get $E - D + 9D \equiv 0 \pmod 7$, or $E + D \equiv 0 \pmod 7$. Now
add the first and second relations for any particular value of $i$
and we get $(f_{i} - d_{i})10^{9-i} + E + D \equiv 0 \pmod 7$. But we
know $E+D$ is divisible by 7, and 10 is coprime to 7, so $d_{i} -
f_{i} \equiv 0 \pmod 7$.

\item[A--4]
Let $s_{k} = x_{1} + \cdots + x_{k} - k(n-1)/n$, so that $s_{n} =
s_{0} = 0$. These form a cyclic sequence that doesn't change when you
rotate the necklace, except that the entire sequence gets translated
by a constant. In particular, it makes sense to choose $x_{i}$ for
which $s_{i}$ is maximum and make that one $x_{n}$; this way $s_{i}
\leq 0$ for all $i$, which gives $x_{1} + \cdots + x_{i} \leq
i(n-1)/n$, but the right side may be replaced by $i-1$ since the left
side is an integer.

\item[A--5]
Everyone (presumably) knows that the set of solutions of a system of
linear first-order differential equations with constant coefficients
is $n$-dimensional, with basis vectors of the form $f_{i}(t)
\vec{v}_{i}$ (i.e.\ a function times a constant vector), where the
$\vec{v}_{i}$ are linearly independent. In
particular, our solution $\vec{x}(t)$ can be written as $\sum_{i=1}^{n}
c_{i}f_{i}(t) \vec{v}_{1}$.

Choose a vector $\vec{w}$ orthogonal to $\vec{v}_{2}, \dots,
\vec{v}_{n}$ but not to $\vec{v}_1$. Since $\vec{x}(t) \to 0$ as $t
\to \infty$, the same is true of $\vec{w} \cdot \vec{x}$; but that is
simply $(\vec{w} \cdot \vec{v}_{1}) c_{1} f_{1}(t)$. In other words,
if $c_{i} \neq 0$, then $f_{i}(t)$ must also go to 0.

However, it is easy to exhibit a solution which does not go to 0. The
sum of the eigenvalues of the matrix $A = (a_{ij})$, also known as the
trace of $A$, being the sum of the diagonal entries of $A$, is
nonnegative, so $A$ has an eigenvalue $\lambda$ with nonnegative real
part, and a corresponding eigenvector $\vec{v}$. Then $e^{\lambda t}
\vec{v}$ is a solution that does not go to 0. (If $\lambda$ is not
real, add this solution to its complex conjugate to get a real
solution, which still doesn't go to 0.)

Hence one of the $c_{i}$, say $c_{1}$, is zero, in which case
$\vec{x}(t) \cdot \vec{w} = 0$ for all $t$.

\item[A--6]
View this as a random walk/Markov process with states $(i,j,k)$ the
triples of integers with sum 0, corresponding to the difference
between the first, second and third rows with their average (twice
the number of columns). Adding a new column adds on a random
permutation of the vector $(1,0,-1)$. I prefer to identify the
triple $(i,j,k)$ with the point $(i-j) + (j-k)\omega +
(k-i)\omega^{2}$ in the plane, where $\omega$ is a cube root of
unity. Then adding a new column corresponds to moving to one of the
six neighbors of the current position in a triangular lattice.

What we'd like to argue is that for large enough $n$, the ratio of
the probabilities of being in any two particular states goes to 1.
Then in fact, we'll see that eventually, about six times as many
matrices have $a=b-1,b=c-1$ than $a=b=c$. This is a pain to prove,
though, and in fact is way more than we actually need.

Let $C_{n}$ and $A_{n}$ be the probability that we are at the origin,
or at a particular point adjacent to the origin, respectively. Then
$C_{n+1} = A_{n}$. (In fact, $C_{n+1}$ is $1/6$ times the sum of the
probabilities of being at each neighbor of the origin at time $n$, but
these are all $A_{n}$.) So the desired result, which is that
$C_{n}/A_{n} \geq 2/3$ for some large $n$, is equivalent to
$A_{n+1}/A_{n} \geq 2/3$.

Suppose on the contrary that this is not the case; then $A_{n} < c
(2/3)^{n}$ for some constant $n$. However, if $n=6m$, the probability
that we chose each of the six types of moves $m$ times is already
$(6m)!/[m!^{6} 6^{6m}]$, which by Stirling's approximation is
asymptotic to a constant times $m^{-5/2}$. This term alone is bigger
than $c (2/3)^{n}$, so we must have $A_{n+1}/A_{n} \geq 2/3$ for
some $n$. (In fact, we must have $A_{n+1}/A_{n} \geq 1-\epsilon$ for
any $\epsilon>0$.)

\item[B--1]  For a given $\pi$, no more than three different values of $\pi(x)$
are possible (four would require one part each of size at least
1,2,3,4, and that's already more than 9 elements). If no such $x, y$
exist, each pair $(\pi(x), \pi'(x))$ occurs for at most 1 element of
$x$, and
since there are only $3 \times 3$ possible pairs, each must occur
exactly once. In particular, each value of $\pi(x)$ must occur 3
times. However, clearly any given value of $\pi(x)$ occurs $k\pi(x)$
times, where $k$ is the number of distinct partitions of that size.
Thus $\pi(x)$ can occur 3 times only if it equals 1 or 3, but we have
three distinct values for which it occurs, contradiction.

\item[B--2]
For those who haven't taken enough physics, ``rolling without
slipping'' means that the perimeter of the ellipse and the curve pass
at the same rate, so all we're saying is that the perimeter of the
ellipse equals the length of one period of the sine curve. So set up
the integrals:
\begin{multline*}
\int_{0}^{2\pi} \sqrt{(-a \sin \theta)^{2} + (b \cos \theta)^{2}}\,
d\theta\\
 = \int_{0}^{2\pi a} \sqrt{1 + (c/a \cos x/a)^{2}}\,dx.
\end{multline*}
Let $\theta = x/a$ in the second integral and write 1 as $\sin^{2}
\theta + \cos^{2} \theta$ and you get
\begin{multline*}
\int_{0}^{2\pi} \sqrt{a^{2} \sin^{2} \theta + b^{2} \cos^{2}
\theta}\,d\theta\\
 = \int_{0}^{2\pi} \sqrt{a^{2} \sin^{2} \theta +
(a^{2} + c^{2}) \cos^{2} \theta}\,d\theta.
\end{multline*}
Since the left side is increasing as a function of $b$, we have
equality if and only if $b^{2} = a^{2} + c^{2}$.

\item[B--3]
For $n=1$ we obviously get 45, while for $n=3$ the answer is 0
because it both changes sign (because determinants are alternating)
and remains unchanged (by symmetry) when you switch any two rows other
than the first one. So only $n=2$ is left. By the multilinearity of
the determinant, the answer is the determinant of the matrix whose
first (resp. second) row is the sum of all possible first (resp.
second) rows. There are 90 first rows whose sum is the vector $(450,
405)$, and 100 second rows whose sum is $(450, 450)$. Thus the answer
is $450\times 450 - 450 \times 405 = 45 \times 450 = 20250.$

\item[B--4]
The infinite continued fraction is defined as the limit of the
sequence $L_{0} = 2207, L_{n+1} = 2207-1/L_{n}$. Notice that the
sequence is strictly decreasing (by induction) and thus indeed has a
limit $L$, which satisfies $L = 2207 - 1/L$, or rewriting, $L^{2} -
2207L + 1 = 0$. Moreover, we want the greater of the two roots.

Now how to compute the eighth root of $L$? Notice that if $x$
satisfies the quadratic $x^{2} - ax + 1 = 0$, then we have
\begin{align*}
0 &= (x^{2} - ax + 1)(x^{2} + ax + 1) \\
&= x^{4} - (a^{2} - 2)x^{2} + 1.
\end{align*}
Clearly, then, the positive square roots of the quadratic $x^{2} -
bx + 1$ satisfy the quadratic $x^{2} - (b^{2}+2)^{1/2}x + 1 = 0$. Thus
we compute that $L^{1/2}$ is the greater root of $x^{2} - 47x + 1 =
0$, $L^{1/4}$ is the greater root of $x^{2} - 7x+ 1 =0$, and
$L^{1/8}$ is the greater root of $x^{2} - 3x + 1 = 0$, otherwise
known as $(3 + \sqrt{5})/2$.

\item[B--5]
This problem is dumb if you know the Sprague-Grundy theory of normal
impartial games (see Conway, Berlekamp and Guy, {\it Winning Ways},
for details). I'll describe how it applies here. To each position you
assign a {\em nim-value} as follows. A position with no moves (in
which case the person to move has just lost) takes value 0. Any other
position is assigned the smallest number not assigned to a valid move
from that position.

For a single pile, one sees that an empty pile has value 0, a pile of
2 has value 1, a pile of 3 has value 2, a pile of 4 has value 0, a
pile of 5 has value 1, and a pile of 6 has value 0.

You add piles just like in standard Nim: the nim-value of the
composite of two games (where at every turn you pick a game and make
a move there) is the ``base 2 addition without carries'' (i.e.\
exclusive OR) of the nim-values of the constituents. So our starting
position, with piles of 3, 4, 5, 6, has nim-value $2 \oplus 0 \oplus
1 \oplus 0 = 3$.

A position is a win for the player to move if and only if it has a
nonzero value, in which case the winning strategy is to always move to
a 0 position. (This is always possible from a nonzero position and
never from a zero position, which is precisely the condition that
defines the set of winning positions.) In this case, the winning move
is to reduce the pile of 3 down to 2, and you can easily describe the
entire strategy if you so desire.

\item[B--6]
Obviously $\alpha, \beta, \gamma$ have to be greater than 1, and no
two can both be rational, so without loss of generality assume that
$\alpha$ and $\beta$ are irrational.
Let $\{x\} = x - \lfloor x \rfloor$ denote the fractional part of $x$.
Then $m \in S(\alpha)$ if and only if $f(m/\alpha) \in (1-1/\alpha,1)
\cup \{0\}$. In particular, this means that $S(\alpha) \cap \{1,
\dots, n\}$ contains $\lceil (n+1)/\alpha \rceil -1$ elements, and
similarly. Hence for every integer $n$,
\[
n = \left\lceil \frac{n+1}\alpha \right\rceil +
 \left\lceil \frac{n+1}\beta \right\rceil +
 \left\lceil \frac{n+1}\gamma \right\rceil -3.
\]
Dividing through by $n$ and taking the limit as $n \to \infty$ shows
that $1/\alpha + 1/\beta + 1/\gamma = 1$. That in turn implies that
for all $n$,
\[
\left\{ - \frac{n+1}{\alpha} \right\} +
\left\{ - \frac{n+1}{\beta} \right\} +
\left\{ - \frac{n+1}{\gamma} \right\} = 2.
\]
Our desired contradiction is equivalent to showing that the left side actually
takes the value 1 for some $n$. Since the left side is
an integer, it suffices to show that $\{ -(n+1)/\alpha\} +
\{-(n+1)/\beta\} < 1$ for some $n$.

A result in ergodic theory (the two-dimensional version of the Weil
equidistribution theorem) states that if $1,r,s$ are linearly
independent over the rationals, then the set of points $(\{nr\},
\{ns\}$ is dense (and in fact equidistributed) in the unit square. In
particular, our claim definitely holds unless $a/\alpha + b/\beta =
c$ for some integers $a,b,c$.

On the other hand, suppose that such a relation does hold. Since
$\alpha$ and $\beta$ are irrational, by the one-dimensional Weil
theorem, the set of points $(\{-n/\alpha\}, \{-n/\beta\}$ is dense in
the set of $(x,y)$ in the unit square such that $ax + by$ is an integer.
It is simple enough to show that this set meets the region $\{(x,y)
\in [0,1]^{2}: x+y<1\}$ unless $a+b$ is an integer, and that would
imply that $1/\alpha + 1/\beta$, a quantity between 0 and 1, is an
integer. We have our desired contradiction.

\end{itemize}
\end{document}
