\documentclass[amssymb,twocolumn,pra,10pt,aps,nofootinbib]{revtex4-1}
\usepackage{mathptmx,amsmath, multirow}

\begin{document}
\title{The 83rd William Lowell Putnam Mathematical Competition \\
    Saturday, December 3, 2022}
\maketitle

\begin{itemize}

\item[A1]
Determine all ordered pairs of real numbers $(a,b)$ such that the line $y = ax+b$ intersects the curve $y = \ln(1+x^2)$ in exactly one point.

\item[A2]
Let $n$ be an integer with $n \geq 2$. Over all real polynomials $p(x)$ of degree $n$, what is the largest possible number of negative coefficients of $p(x)^2$?

\item[A3]
Let $p$ be a prime number greater than 5. Let $f(p)$ denote the number of infinite sequences $a_1, a_2, a_3, \dots$ such that
$a_n \in \{1, 2, \dots, p-1\}$ and $a_n a_{n+2} \equiv 1 + a_{n+1} \pmod{p}$ for all $n \geq 1$. Prove that $f(p)$ is congruent to 0 or 2 $\pmod{5}$.

\item[A4]
Suppose that $X_1, X_2, \dots$ are real numbers between 0 and 1 that are chosen independently and uniformly at random. Let $S = \sum_{i=1}^k X_i/2^i$, where $k$ is the least positive integer such that $X_k < X_{k+1}$, or $k = \infty$ if there is no such integer. Find the expected value of $S$.
 
\item[A5]
Alice and Bob play a game on a board consisting of one row of 2022 consecutive squares. They take turns placing tiles that cover two adjacent squares, with Alice going first. By rule, a tile must not cover a square that is already covered by another tile. The game ends when no tile can be placed according to this rule. Alice's goal is to maximize the number of uncovered squares when the game ends; Bob's goal is to minimize it. What is the greatest number of uncovered squares that Alice can ensure at the end of the game, no matter how Bob plays?

\item[A6]
Let $n$ be a positive integer. Determine, in terms of $n$, the largest integer $m$ with the following property: There exist real numbers $x_1,\dots,x_{2n}$ with $-1 < x_1 < x_2 < \cdots < x_{2n} < 1$ such that the sum of the lengths of the $n$ intervals
\[
[x_1^{2k-1}, x_2^{2k-1}], [x_3^{2k-1},x_4^{2k-1}], \dots, [x_{2n-1}^{2k-1}, x_{2n}^{2k-1}]
\]
is equal to 1 for all integers $k$ with $1 \leq k \leq m$.

\item[B1]
Suppose that $P(x) = a_1 x + a_2 x^2 + \cdots + a_n x^n$ is a polynomial with integer coefficients, with $a_1$ odd. Suppose that $e^{P(x)} = b_0 + b_1 x + b_2 x^2 + \cdots$ for all $x$. Prove that $b_k$ is nonzero for all $k \geq 0$.

\item[B2]
Let $\times$ represent the cross product in $\mathbb{R}^3$. For what positive integers $n$ does there exist a set $S \subset \mathbb{R}^3$ with exactly $n$ elements such that 
\[
S = \{v \times w: v, w \in S\}?
\]

\item[B3]
Assign to each positive real number a color, either red or blue. Let $D$ be the set of all distances $d > 0$ such that there are two points of the same color at distance $d$ apart. Recolor the positive reals so that the numbers in $D$ are red and the numbers not in $D$ are blue. If we iterate this recoloring process, will we always end up with all the numbers red after a finite number of steps?

\smallskip

\item[B4]
Find all integers $n$ with $n \geq 4$ for which there exists a sequence of distinct real numbers $x_1,\dots,x_n$ such that each of the sets
\begin{gather*}
\{x_1,x_2,x_3\}, \{x_2,x_3,x_4\}, \dots, \\
\{x_{n-2},x_{n-1},x_n\}, \{x_{n-1},x_n, x_1\}, \mbox{ and } \{x_n, x_1, x_2\}
\end{gather*}
forms a 3-term arithmetic progression when arranged in increasing order.

\item[B5]
For $0 \leq p \leq 1/2$, let $X_1, X_2, \dots$ be independent random variables such that
\[
X_i = \begin{cases} 1 & \mbox{with probability $p$,} \\
-1 & \mbox{with probability $p$,} \\
0 & \mbox{with probability $1-2p$,}
\end{cases}
\]
for all $i \geq 1$. Given a positive integer $n$ and integers $b, a_1, \dots, a_n$, let $P(b, a_1, \dots, a_n)$ denote the probability that $a_1 X_1 + \cdots + a_n X_n = b$. For which values of $p$ is it the case that
\[
P(0, a_1, \dots, a_n) \geq P(b, a_1, \dots, a_n)
\]
for all positive integers $n$ and all integers $b, a_1, \dots, a_n$?

\item[B6]
Find all continuous functions $f: \mathbb{R}^+ \to \mathbb{R}^+$ such that
\[
f(xf(y)) + f(yf(x)) = 1 + f(x+y)
\]
for all $x,y > 0$.
\end{itemize}

\end{document}
